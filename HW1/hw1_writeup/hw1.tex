\documentclass[12pt, letterpaper]{article}
\usepackage[utf8]{inputenc}
\usepackage[english]{babel} % To obtain English text with the blindtext package
\usepackage{blindtext}
\usepackage{amsmath}
\usepackage{amssymb}
\usepackage{enumitem}
\usepackage{courier}
\usepackage{listings}

\lstdefinestyle{mystyle}{
    basicstyle=\ttfamily\footnotesize,
    breakatwhitespace=false,         
    breaklines=true,                 
    captionpos=b,                    
    keepspaces=true,                 
    numbers=left,                    
    numbersep=5pt,                  
    showspaces=false,                
    showstringspaces=false,
    showtabs=false,                  
    tabsize=2
}

\lstset{style=mystyle}

\title{CSCI2291 Homework 1}
\author{Jack Moffatt}
\date{January 27, 2022}


\begin{document}
\maketitle
\noindent\makebox[\linewidth]{\rule{18cm}{0.4pt}}

\section*{Problem 1}
\begin{enumerate}[label=(\alph*)]
    \item Our code reads as follows 
    \[
    \texttt{min([x ** 2 for x in L])}
    \]
    Our list in this case is constructed as a for statement that creates a list of the squared 
    elements in $L$. Then, after this list has been constructed, the $min$ function takes 
    the smallest element of this list of squares. Using the sample list 
    \[
        \texttt{L = [4, -4, 6, 8, -2, 7]}
    \]  
    our code gives us the answer, $4$, which is the square of $-2$.
    \item Our code reads as follows 
    \[ 
    \texttt{D[-3]}
    \]
    This code uses negative indexes. $D[-1]$ would return this last element in the list. 
    Likewise, $D[-2]$ would return the firs-from-last element, so it follows that 
    $D[-3]$ returns the second-from-last element. Using the sample NumPy array:
    \[
        \texttt{D = np.array([4, -4, 6, 8, -2, 7])}
    \]  
    our code gives us the answer $8$, which is the second to last element in the NumPy array.
    \item Our code reads as follows
    \[ 
    \texttt{sum([n**3 for n in range(-50, (10**4 + 1))])}
    \]
    First we construct a list of all the terms in the series with a for statement. The lower bound in the 
    $range$ function is inclusive, so we use -50. But the upper bound is exclusive, so we must use $\texttt{(10 ** 4 + 1)}$, 
    so that the last term iterated over is $\texttt{10 ** 4}$. Then for each term in this range we take its cube and put it into a list, 
    then we take the sum of the elements in the list with the $\texttt{sum()}$ function. 
    When executed our code returns the answer: 
    \[
        2498500323344376 
    \] 
    \item Our code reads as follows
    \[
    \texttt{[n for n in range(100) if n**4 > 500 * n]}
    \]
    First, we iterate over the range of $[0, 99]$, by using $\texttt{range(100)}$. Then for each element 
    we iterate over, we apend it to our list if it satisfies the condition $\texttt{(n ** 4 > 500 * n)}$. When executed 
    our code gives us a list 
    \[
        \texttt{[8, 9, 10, ... 99]}
    \]  
    \item Our code reads as follows
    \[ 
    \texttt{len([ D[key] for key in D if D[key] != 'red'])}
    \]
    In this expression, we create a list of dictionary values for each value in our dictionary which is not 
    $\texttt{'red'}$. Then once we have a list of every value in our dictionary that is not $\texttt{'red'}$, we 
    use the $\texttt{len()}$ function to get the number of elements in our list. Using the example dictionary
\begin{lstlisting}
D = { 1 : 'red', 
      2 : 'black', 
      3 : 'red', 
      4 : 'black', 
      5 : 'red', 
      6 : 'black', 
      7 : 'red'}
\end{lstlisting}
    Executing our code, we get the answer 3. 
\end{enumerate}
\newpage

\noindent\makebox[\linewidth]{\rule{18cm}{0.4pt}}
\section*{Problem 2}
First let us take the integral of $f'(x)$:
\begin{align*}
    f(x) &= \int f'(x) dx \\
    f(x) &= \int 1 - sin(x) dx \\
    f(x) &= x + cos(x) + C
\end{align*}
Now, let us solve for $C$. We know that $f(\pi) = 1$, so we compute
\begin{align*}
    f(\pi) &= \pi + cos(\pi) + C \\
    1 &= \pi - 1 + C \\
    C &= 2 - \pi
\end{align*}
So we find that $f(x) = x + cos(x) + 2 - \pi$. Finally, we can solve for $f(0)$. We compute
\begin{align*}
    f(0) &= 0 + cos(0) + 2 - \pi \\
    f(0) &= 1 + 2 - \pi \\
    f(0) &= 3 - \pi
\end{align*}


\noindent\makebox[\linewidth]{\rule{18cm}{0.4pt}}
\section*{Problem 3}
\begin{enumerate}[label=(\alph*)]
    \item We will define a Python function that generates the largest integer, $m$ 
        satisfying the given inequality. To do this, we can initialize $m = 1$ and then create a while loop 
        which increments by $1$ with each iteration. Our code reads as follows: 
\begin{lstlisting}[language = Python]
def generate_largest_a(): 
    m = 1
    while 10 ** m < (1000 * (m ** 6)):
        m += 1
    m -= 1
    print(f"m = {m}") 
\end{lstlisting}
    When this function is called, the while loop will begin by checking the statement in the header
    with $m = 1$. After finding it is true, the loop will increment $m$ by 1 and then jump back to the header, where the process
    will repeat with $m=2$. It will continue to repeat the process until $m=8$. \\ \\At this point the header will be satisfied, but then when $m$ increments
    to 9, the while loop header will be checked again and found to be false, which terminates the loop. Since at this point $m=9$ but $9$ does not satisfy the equality 
    we must decrement $m$ back to 8. And then we may print $m = 8$ as our final answer. \\ \\
    This answer can also be checked by hand if we compute 
    \begin{align*}
        10^8 &= 100000000 \\
        1000 \cdot 8^6 &= 262144000 \\
        \implies 10^8 &< 1000 \cdot 8^6 \\
    \end{align*}
    while, 
    \begin{align*}
        10^9 &= 1000000000 \\
        1000 \cdot 9^6 &= 531441000 \\
        \implies 10^9 &> 1000 \cdot 9^6 \\
    \end{align*}
    So we confirm that $m=9$ does not satisfy the inequality, thus our Python code correctly tells us that 
    the largest integer satisfying the inequality is 
    \[
        m = 8
    \]  

    \item We know that for $m = 100$, the cubed term will be growing far faster than our squared and linear 
        terms, so by the time we reach $m = 100$, our function will be negative and will coninute to decrease as $m\rightarrow \infty$.
        This is due to the relative growth rate of the functions $x, x^2, x^3$. \\
        As a result, if we begin iteration from $m = 100$ and decrement $m$ until our condition is satisfied,
        we can quickly find the largest such $m$ that satisfies the inequality. Our code reads as follows:
\begin{lstlisting}[language = Python]
def generate_largest_b():
m = 100
while 1000 - (100 * m) + (5 * m ** 2) - ((1 / 15) * (m ** 3)) < 500:
    m -= 1
print(m)
\end{lstlisting}
    When running this code, we see that we are given the answer $m = 45$. We can verify this:
    \begin{align*} 
        1000 - 100(45) + 5(45^2) -frac{(45^3)}{15} &= 550 \\
        1000 - 100(46) + 5(46^2) -frac{(46^3)}{15} &= 490.933 
    \end{align*}
    We see that with $m=45$ the function is greater than 500, while if we set $m$ to 46, 
    then the function is less than 500. So our python code helps us find the answer of 
    \[
        m = 45    
    \]
\end{enumerate}





\end{document}